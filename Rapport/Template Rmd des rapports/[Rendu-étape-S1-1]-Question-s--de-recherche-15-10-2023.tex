\documentclass[mstat,12pt]{unswthesis}



%%%%%%%%%%%%%%%%%%%%%%%%%%%%%%%%%%%%%%%%%%%%%%%%%%%%%%%%%%%%%%%%%%
% 
% OK...Now we get to some actual input.  The first part sets up
% the title etc that will appear on the front page
%
%%%%%%%%%%%%%%%%%%%%%%%%%%%%%%%%%%%%%%%%%%%%%%%%%%%%%%%%%%%%%%%%%

\title{Projet TER par l'équipe 5\\[0.5cm]Rapport intermédiaire 1 -
Questions de recherche}

\authornameonly{Thomas AYRIVIÉ, Mehdi BELKHITER, Jamila CHERKAOUI, Dina
EL HIJJAWI, Magatte LO. }

\author{\Authornameonly}

\copyrightfalse
\figurespagefalse
\tablespagefalse

%%%%%%%%%%%%%%%%%%%%%%%%%%%%%%%%%%%%%%%%%%%%%%%%%%%%%%%%%%%%%%%%%
%
%  And now the document begins
%  The \beforepreface and \afterpreface commands puts the
%  contents page etc in
%
%%%%%%%%%%%%%%%%%%%%%%%%%%%%%%%%%%%%%%%%%%%%%%%%%%%%%%%%%%%%%%%%%%


\input{header.tex}

\renewcommand{\contentsname}{Table des matières}

\renewcommand{\chaptername}{Chapitre}



\begin{document}

\beforepreface

%\afterpage{\blankpage}

% plagiarism

\prefacesection{Déclaration de non plagiat}

\vskip 2pc \noindent Nous déclarons que ce rapport est le fruit de notre seul travail, à part lorsque cela est indiqué  explicitement. 

\vskip 2pc  \noindent Nous acceptons que la personne évaluant ce rapport puisse, pour les besoins de cette évaluation:
\begin{itemize}
\item la reproduire et en fournir une copie à un autre membre de l'université; et/ou,
\item en communiquer une copie à un service en ligne de détection de plagiat (qui pourra en retenir une copie pour les besoins d'évaluation future).
\end{itemize}

\vskip 2pc \noindent Nous certifions que nous avons lu et compris les règles ci-dessus.\vspace{24pt}

\vskip 2pc \noindent En signant cette déclaration, nous acceptons ce qui précède.
\vskip 2pc \noindent
Signature : \textbf{\textit{Thomas AYRIVIÉ,
n°22000580}},   Date : \textbf{\textit{15/10/2023}}. \\[1cm]
Signature : \textbf{\textit{Mehdi BELKHITER,
n°21813356}},   Date : \textbf{\textit{15/10/2023}}. \\[1cm]
Signature : \textbf{\textit{Jamila CHERKAOUI,
n°22309204}},   Date : \textbf{\textit{15/10/2023}}. \\[1cm]
Signature : \textbf{\textit{Dina EL HIJJAWI,
n°22310171}},   Date : \textbf{\textit{15/10/2023}}. \\[1cm]
Signature : \textbf{\textit{Magatte LO,
n°22311161}},   Date : \textbf{\textit{15/10/2023}}. \\[1cm]
\vskip 1pc

%\afterpage{\blankpage}

% Acknowledgements are optional


%%\prefacesection{Remerciements}

%%{\bigskip}

%%{\bigskip\bigskip\bigskip\noindent} 15/10/2023.

%\afterpage{\blankpage}

% Abstract

\prefacesection{Préface}

Dans le cadre de notre master 1 Miashs, les étudiants réalisent un
\textbf{projet de Travaux d’Études et de Recherche} (TER) en lien avec
un commanditaire. Ce présent document est un rapport intermédiaire dans
lequel nous présenterons la définition de notre projet et nos questions
de recherches regroupées dans quatre parties
générales.\\[1cm] \textbf{Sujet 6 :} « Identification de voies de
signalisation activées par des traitements de cancer du sein
».\\[1cm] \textbf{Encadrement pédagogique :} \newline Mme Sophie Lèbre,
\href{mailto:sophie.lebre@univ-montp3.fr}{\nolinkurl{sophie.lebre@univ-montp3.fr}}
(Institut Montpelliérain Alexander Grothendieck) avec Mathilde Robin
(Institut de Recherche en Cancérologie de Montpellier, LIRMM Laboratoire
d'informatique, de robotique et de microélectronique de Montpellier),
Charles Lecellier (LIRMM) et Laurent Bréhélin
(LIRMM).''\\[1cm] \textbf{Composition du groupe :} \newline Dina EL
HIJJAWI, n°22310171,
\href{mailto:dina.el-hijjawi@etu.univ-montp3.fr}{\nolinkurl{dina.el-hijjawi@etu.univ-montp3.fr}}.
Coordinatrice. \newline Thomas AYRIVIÉ, n°22000580,
\href{mailto:thomas.ayrivie@etu.univ-montp3.fr}{\nolinkurl{thomas.ayrivie@etu.univ-montp3.fr}}.
\newline Mehdi BELKHITER, n°21813356,
\href{mailto:mehidi.belkhiter@etu.univ-montp3.fr}{\nolinkurl{mehidi.belkhiter@etu.univ-montp3.fr}}.
\newline Jamila CHERKAOUI, n°22309204,
\href{mailto:jamila.cherkaoui@etu.univ-montp3.fr}{\nolinkurl{jamila.cherkaoui@etu.univ-montp3.fr}}.
\newline Magatte LO, n°22311161,
\href{mailto:magatte.lo@etu.univ-montp3.fr}{\nolinkurl{magatte.lo@etu.univ-montp3.fr}}.\\[1cm] 

%\afterpage{\blankpage}


\afterpreface





%%%%%%%%%%%%%%%%%%%%%%%%%%%%%%%%%%%%%%%%%%%%%%%%%%%%%%%%%%%%%%%%%%
%
% Now we can start on the first chapter
% Within chapters we have sections, subsections and so forth
%
%%%%%%%%%%%%%%%%%%%%%%%%%%%%%%%%%%%%%%%%%%%%%%%%%%%%%%%%%%%%%%%%%%



%%%%%%%%%%%%%%%%%%%%%%%%%%%%%%%%%%%%%

%\afterpage{\blankpage}


\hypertarget{contexte}{%
\chapter{Contexte}\label{contexte}}

\hypertarget{notre-commanditaire}{%
\section{Notre commanditaire}\label{notre-commanditaire}}

Le commanditaire de projet est une collaboration avec Mathilde Robin qui
est ingénieure à l'IRCM ainsi que Laurent Bréhélin (chercheur CNRS au
LIRMM), Charles Lecellier (chercheur à l'IGMM) et Sophie Lèbre est
maître de conférences au sein du département Mathématiques et
Informatique Appliquées (MIAp) de l'Université Paul Valéry Montpellier
3, membre de l'équipe Probabilité et Statistiques de l'Institut
Montpelliérain Alexander Grothendieck (IMAG), co-responsable du Master
MIASHS (Mathématiques et Informatique pour les Sciences Humaines et
Sociales), master en alternance sur les 2 années M1-M2.

\bigskip

Le sujet du cancer en général fait partie du travail de recherche de Mme
Lebre. Pour elle, l'augmentation du nombre de patients atteints de
cancer, et pas seulement de cancer du sein, est préoccupante, ce sujet
devrait donc intéresser tous ceux qui travaillent dans le domaine de la
recherche ou en dehors.

\bigskip

Au-delà de cet aspect biologique, elle s'intéresse également à la
condition difficile statistique à traiter (avec de fortes corrélations).
Son travail lui permet de comparer différentes méthodes pour la
sélection des variables quand la corrélation est élevée et proposer de
nouvelles méthodes innovantes.

\hypertarget{les-actualituxe9s}{%
\section{Les actualités}\label{les-actualituxe9s}}

Selon les Nations Unies, le cancer est la deuxième cause de décès dans
le monde entier et a fait 9,6 millions de morts en 2018, soit un décès
sur
six\footnote{ ONU, Cancer. https://www.who.int/fr/health-topics/cancer}.
La recherche sur le cancer revêt donc une importance capitale pour les
chercheurs et la société. Elle vise à réduire la mortalité en
développant de nouvelles méthodes de dépistage et de traitement,
améliore la qualité de vie des patients, permet une meilleure
compréhension des causes sous-jacentes du cancer, stimule l'innovation
médicale, réduit les coûts des soins de santé, lutte contre les
disparités en matière de santé, favorise l'innovation et aide à prévenir
les cancers évitables. En somme, elle contribue de manière significative
à la santé, au bien-être et à l'économie, faisant d'elle une priorité
majeure pour la société.

\bigskip

Le cancer du sein représente un défi de santé mondial majeur, touchant
un grand nombre de personnes à l'échelle internationale. Au sein de
LIRMM, les chercheurs mènent des recherches sur les différents types de
cancer mais le jeu de données que nous devons exploiter dans ce projet
est spécifiquement axé sur le cancer du sein. Cette décision est motivée
par la fréquence élevée de cette maladie, avec un nombre alarmant de 61
214 nouveaux cas enregistrés en 2023, selon les données de l'Institut
national du
cancer\footnote{ Institut National du Cancer, le cancer du sein. https://www.e-cancer.fr/Professionnels-de-sante/Les-chiffres-du-cancer-en-France/Epidemiologie-des-cancers/Les-cancers-les-plus-frequents/Cancer-du-sein}.

\bigskip

À l'heure d'écriture de ce rapport intermédiaire, nous sommes au mois
d'octobre. Le mois d'octobre est devenu un mois de sensibilisation au
cancer du sein à l'échelle mondiale, marqué par des campagnes
emblématiques telles qu'Octobre
Rose\footnote{ Ars Bretagne, Octobre Rose. https://www.bretagne.ars.sante.fr/octobre-rose-un-mois-pour-sensibiliser-au-depistage-du-cancer-du-sein-0}.
Ces initiatives de sensibilisation comprennent des récoltes de dons
visant à soutenir la recherche sur le cancer du sein, à sensibiliser le
public et à fournir un soutien aux patients et à leurs familles.

\hypertarget{choix-du-sujet}{%
\section{Choix du sujet}\label{choix-du-sujet}}

Il se trouve donc que le fait de travailler sur une petite partie de ce
grand domaine d'étude peut intéresser davantage certains d'entre nous,
les étudiants qui y travaillent, et les motiver à poursuivre la
recherche à l'avenir, en s'ajoutant à la multitude de chercheurs qui
tentent d'approfondir leur compréhension du cancer et de la manière de
le traiter.

\hypertarget{duxe9finition-du-sujet}{%
\chapter{Définition du sujet}\label{duxe9finition-du-sujet}}

Le sujet de recherche se concentre sur l'identification des voies de
signalisation activées en réponse aux traitements du cancer du sein. Au
fil du temps, les cellules cancéreuses ont la capacité de s'adapter et
de développer des mécanismes de résistance, y compris face à la
chimiothérapie. Comprendre ces mécanismes est essentiel pour améliorer
les stratégies de traitement.

\bigskip

Pour ce faire, nous utilisons une analyse approfondie des séquences
génétiques afin de déterminer quelles combinaisons de nucléotides
obtiennent les scores les plus élevés dans les modèles de prédiction de
la classe Y. Ces scores reflètent la similitude ou la pertinence d'une
séquence donnée par rapport à un modèle de référence.

\bigskip

Préalablement, un modèle de régression a été développé pour expliquer
l'activité des gènes en utilisant uniquement la séquence
d'ADN\footnote{ Hsiaowang Chen, Series GSE130787 de NCBI. \newline https://www.ncbi.nlm.nih.gov/geo/query/acc.cgi?acc=GSE130787}.
Cependant, nous visons maintenant à mettre en place un nouveau modèle de
classification, pour qu'on puisse construire un modèle capable de
prédire quels gènes sont différentiellement exprimés (soit actifs soit
inhibés) en réponse à un traitement donné. Les détails seront abordés
ultérieurement au cours du projet.

\hypertarget{donnuxe9es-et-traitements}{%
\chapter{Données et traitements}\label{donnuxe9es-et-traitements}}

Pour mener à bien notre recherche, nous utilisons la base de données
Jaspar\footnote{ Wasserman Lab. http://www.cmmt.ubc.ca/wasserman-lab/},
une ressource spécialisée dans les matrices de séquences d'ADN liées aux
sites de liaison des facteurs de transcription. Ces matrices, également
appelées matrices de poids de position (PWM), offrent une représentation
numérique des motifs de liaison des facteurs de transcription qui sont
des protéines codées par des gènes ADN qui sont responsables de
l'activation ou l'inhibition d'autres gènes. Jaspar fournit des
informations détaillées sur la fréquence de chaque base (A, C, G,
T)\footnote{ Les bases azotées symbolisées par A, C, G, T revoient dans le mémoire à A pour Adénine, C pour Cytosine, G pour Guanine, T pour Thymine.}
à chaque position au sein de ces motifs de liaison. Cette base de
données est essentielle pour développer des modèles plus précis dans le
but de mieux comprendre la régulation génique et les réponses
cellulaires aux traitements du cancer du sein et elle est aussi
essentielle pour identifier des sites de liaisons potentiels au sein des
séquences et construire un ensemble de variables explicatives à partir
des données existantes afin d'améliorer la performance et la capacité
prédictive du modèle.

\bigskip

En complément, nous utilisons pareillement la base de données Series
GSE130787
GEO\footnote{ Hsiaowang Chen, Series GSE130787 de NCBI. \newline https://www.ncbi.nlm.nih.gov/geo/query/acc.cgi?acc=GSE130787}
(Gene Expression Omnibus) qui est chargée dans RStudio grâce à
GEOquery\footnote{ Davis S, Meltzer P (2007). “GEOquery: a bridge between the Gene Expression Omnibus (GEO) and BioConductor.” Bioinformatics, 14, 1846–1847. https://bioconductor.org/packages/release/bioc/html/GEOquery.html}.
GEO stocke une grande variété de données d'expression génique provenant
de diverses expériences et technologies, incluant les puces à ADN et les
séquences ARN. Les chercheurs déposent leurs données d'expression
génique dans GEO, permettant ainsi à d'autres scientifiques d'y accéder,
de les partager et de les analyser. Cette base de données facilite
l'exploration et la comparaison des profils d'expression génique dans
différents contextes biologiques et expérimentaux.

\bigskip

Jaspar nous aide à analyser les motifs de liaison des facteurs de
transcription, tandis que Geoquery nous permet d'accéder à une vaste
gamme de données d'expression génique pour une exploration approfondie
des mécanismes biologiques et des réponses aux traitements.

\hypertarget{muxe9thodologie}{%
\chapter{Méthodologie}\label{muxe9thodologie}}

Le but principal est de déterminer les variables, ou motifs, qui jouent
un rôle discriminant entre divers gènes, les classifiant en catégories
telles qu' actifs, inactifs et inhibés. L'objectif du projet est aussi
de décrypter les facteurs de transcription impliqués dans le
développement du cancer du sein . Nous cherchons également à approfondir
notre compréhension du fonctionnement du traitement, en mettant l'accent
sur les cellules qui réagissent le mieux à ce traitement et à identifier
les individus pour lesquels ce traitement est le plus efficace. Sur le
plan statistique, notre démarche implique de comparer différentes
méthodes statistiques pour la sélection de variables, surtout lorsque
deux variables sont corrélées. Cela vise à élucider les nuances entre
ces méthodes et à déterminer la plus adaptée pour notre analyse,
assurant ainsi la précision de nos conclusions dans cette étude
complexe.

\bigskip

Pour constituer un jeu de données, nous utilisons Fimo de la suite Meme
qui sert à rechercher les occurrences individuelles de motifs dans les
séquences. Les données Jaspar y sont intégrées avec les données de
séquences des régions promotrices des
gènes\footnote{ LÈBRE Sophie, ROBIN Mathilde. https://upvdrive.univ-montp3.fr/s/kKRzYDzzcXe32dX}
au format fasta. La sortie de ce logiciel calcule les scores maximum
(sous la forme de p-valeur).

\bigskip

Grâce au package Tidyverse de RStudio, ces données sont ensuite
regroupées dans une matrice qui croise les gènes de chaque séquence et
leurs motifs avec des p-valeurs associées pour chaque croisement.

\bigskip

Une fois cette matrice créée, nous pourrons faire une jointure avec les
données GEOquery NCBI (Patientes suivies en cancer du sein). En
choisissant correctement des valeurs seuils et grâce au logarithme de 10
du Fold Change nous pourrons construire les classes Y (up, down,
neutre).

\bigskip

À partir de là, nous serons dans un problème de classification. Nous
devrons rechercher des groupes de variables corrélées pour trouver un
modèle explicatif qui utilise la structure de corrélation des variables.
Pour cela, nous utiliserons le package glmnet et le package
mlgl\footnote{ Grimonprez, Q., Blanck, S., Celisse, A., and Marot, G. (2023). MLGL: An R Package Implementing Correlated Variable Selection by Hierarchical Clustering and Group-Lasso. Journal of Statistical Software, 106(3), 1–33. https://doi.org/10.18637/jss.v106.i03  3 .Meinshausen N. and Bühlmann P (2010) Stability selection Journal of the Royal Statistical Society: Series B (Statistical Methodology) https://doi.org/10.1111/j.1467-9868.2010.00740.x},
modèle
Lasso\footnote{ Tibshirani R. (Lecture, Spring 2017) Sparsity, the Lasso, and Friends. https://upvdrive.univ-montp3.fr/s/jakYqikj52WZ7Ci}
et elastinet. Nous approfondirons les recherches avec les forêts
aléatoires dans Random Forest dans
R\footnote{ Genuer, R., and Poggi, J. (2020). Random Forests with R. Dans Use R ! https://doi.org/10.1007/978-3-030-56485-8}
et en indiquant les forces et faiblesses du modèle Lasso. Pour un modèle
de classification, la précision, le rappel, le F1-score, et l'aire sous
la courbe ROC (AUC) peuvent être utilisés pour évaluer la performance.

\bigskip

Nos variables sont les motifs de séquence (représentés par
motif\_alt\_id) : Ils décrivent les motifs d'ADN qui sont reconnus par
les facteurs de transcription. Les p-valeurs associées à chaque motif :
Elles donnent une mesure de la signification statistique de l'occurrence
d'un motif dans une séquence d'ADN. Les identifiants de séquence
(sequence\_name): Ils identifient de manière unique chaque séquence
d'ADN analysée.

\bigskip

En parallèle de celà, nous allons faire une analyse exploratoire des
données (Analyse des composantes principales, visualisation de la
structure des données, qualification des variables\ldots). Et nous
allons utiliser les packages RStudio avec de petits jeux de données pour
comprendre leur utilisation correcte.

\bigskip

Pour une analyse d'exploration (comme l'ACP), la variance expliquée par
chaque composante principale peut être utilisée comme métrique. D'autres
métriques comme la régression sera également abordée et utiliser et si
on a du temps l'objectif sera la mise en place d'un modèle de machine
learning (pour prédire pour les prochains patients).

\bigskip

\hypertarget{annexe-questions-de-recherche-plus-spuxe9cifiques}{%
\chapter*{Annexe : Questions de recherche plus
spécifiques}\label{annexe-questions-de-recherche-plus-spuxe9cifiques}}
\addcontentsline{toc}{chapter}{Annexe : Questions de recherche plus
spécifiques}

Questions de recherche plus spécifiques qui peuvent être résolues en
utilisant des nombres et du texte.

\begin{enumerate}
\item \textbf{Combien (régression) ?} Quelle est l'importance relative de chaque motif dans la régulation de l'expression génique ? \textit{Quelle est l'importance relative de chaque motif dans la régulation de l'expression génique ?}
\item \textbf{Quelle catégorie (classification) ?} Un motif donné est-il associé à une augmentation, une diminution ou aucune modification de l'expression génique ? \textit{Un motif donné est-il associé à une augmentation, une diminution ou aucune modification de l'expression génique ?}
\item \textbf{Quel groupe (clustering) ?} Existe-t-il des groupes de motifs qui ont des comportements similaires en termes d'effets sur l'expression génique ? \textit{Existe-t-il des groupes de motifs qui ont des comportements similaires en termes d'effets sur l'expression génique ?}
\item \textbf{Est-ce anormal (détection d'anomalies) ?} Y a-t-il des motifs qui se comportent de manière atypique par rapport à d'autres motifs similaires ? \textit{Y a-t-il des motifs qui se comportent de manière atypique par rapport à d'autres motifs similaires ?}
\item \textbf{Quelle option choisir (recommandation) ?} Quels motifs devraient être ciblés pour des études plus approfondies en fonction de leur pertinence pour la régulation de l'expression génique ?  \textit{Quels motifs devraient être ciblés pour des études plus approfondies en fonction de leur pertinence pour la régulation de l'expression génique ?}
\end{enumerate}







\end{document}

